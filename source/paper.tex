\section{Introduction}

My thesis thereto is that there is meaning that goes beyond the literal interpretation of the words themselves, and this meaning begins with relations, such as those that entail mutual dependence and those that convey paradoxical themes. 

In Koji Tanaka’s paper called The Limit of Language in Daoism, Tanaka argues that the Daoists “seem to have noticed that the limit of language and its paradoxical nature cannot be overcome.” (Tanaka) Therefore, what we get is a perhaps nonconventional approach to describing something that words cannot directly express, and therefore the result may be syntactically paradoxical.

As it is said,
\begin{quotation}
    The way is empty, yet use will not drain it.
\end{quotation}

Or rather as Tanaka playfully put it, 
\begin{quotation}
‘Don’t be guided by any guidance!’ is a paradoxical guidance. The only way to be guided by the guidance is by not being guided by it. But not being guided by the guidance, we are guided by it. How can we be guided or not be guided by this guidance? (Tanaka)
\end{quotation}

Furthermore Tanaka later remarks, “They ‘played’ with the paradox throughout the development of Daoism.” Which while unconventional, I cannot help but feel that a ‘playful’ approach to understanding Daoism may be more successful than something that is rigged. Even in the context of Daoism itself, Tanaka says “[this] is not the picture that is painted by the dominant view of Daoism, however.” (Tanaka)

\section{Understanding}

Tao Te Ching is a collection of chapters, where each chapter is say, a paragraph or two, sometimes more, sometimes less. But altogether, each chapter is rather short, and generally unrelated to adjacent chapters. So therefore, there is no serialization, in the sense of chapters building upon prior chapters.

Likewise, in Tao Te Ching the burden of understanding is placed on the reader rather than the author, and therefore just as the burden of understanding rests on the reader, so too, does the burden of translation rest on the translator especially, and therefore there can be subtle differences between translations themselves, let alone across language barriers. 

As Tanaka argued, “In general, the Western philosophers (and Indian philosophers) take language to be representative.” (Tanaka) Who goes on to say that the Western fixation on language is predominately a fixation on semantics. Yet the problem with Tao Te Ching is that it is a series of rather short poems, without any serialization throughout.

Furthermore Tanaka claims that the classical Chinese philosophers were not interested in the semantics of language, but rather in the “pragmatics of language”. Which to me, feels like a tradeoff, that as I have argued, when you reach the limits of the semantics of language, you must make pragmatic tradeoffs, and therefore we have to accept ambiguity if we are to understand the meaning of Taoism. 

For instance the following are the first lines from two different translators,

\begin{quotation}
    The way that can be spoken of is not the constant way.
\end{quotation} 

\begin{quotation}
    The tao that can be described is not the eternal Tao.
\end{quotation}

We can analyze the literal meaning to the above statements, because there seems to be structure saying that there exists two sets: the eternal Tao A, and the descriptive Tao B, both of which are mutually exclusive, they are disjoint.  Yet, what else is meant by such a statement? What else is meant besides the literal meaning of the words themselves? 

Sometimes it feels like the author was conscious of such. 

\begin{quotation}
    The Sages of old were profound and knew the ways of subtlety and discernment.
    Their wisdom is beyond our comprehension.
    Because their knowledge was so far superior I can only give a poor description. 
\end{quotation}

(Which personally speaking should be the first line.)

But in some ways there can be personal gems buried deep within its pages. For instance,
\begin{quotation}
    When you understand all things can you step back from your own understanding? 
\end{quotation}


\section{Relationships and Proportions}

The world depicted in Tao Te Ching is a world defined by mutual dichotomies and mutual interdependence, where something and nothing must necessarily and mutually give rise to each other, just as much as they give rise to themselves, and from this outgrowth, forms our reality.

\begin{quotation}
Look for it, and it can't be seen.
Listen for it, and it can't be heard.
Grasp for it, and it can't be caught.
These three cannot be further described, so we treat them as The One. 

Its highest is not bright.
Its depths are not dark.

Unending, unnamable, it returns to nothingness.
Formless forms, and imageless images, subtle, beyond all understanding. 

Approach it and you will not see a beginning; follow it and there will be no end.
When we grasp the Tao of the ancient ones, we can use it to direct our life today. 
To know the ancient origin of Tao: this is the beginning of wisdom. 
\end{quotation}

This is like the model of our word, and just like any other model, you must describe how the model works. Yet Tao Te Ching seems like it circumvent the issue by avoiding the discrete in favor of indiscrete relations, and from these relations, it builds an axiomatic model that underpins the philosophy.

That is, this philosophy is given in terms of relations. Such as good and evil: good and evil mutually coexist, you cannot have one without the other. They imply and define each other just as much as they define themselves, and therefore good and evil aren’t standalone concepts, but relational concepts, you define one in relation to the other, and vice versa. 

Analogously, when I was a kid I once drew something that was so good I never saw it until the end of the school year when I managed to get it back. From someone who isn’t necessarily known for being good at such, looking back I’ve come to the conclusion that what enabled me to draw such an exceeding good portrait was my unwitting fixation on relations. As I divided the paper into grids, I never fixated on the individual details themselves, but in terms of relations between the details and the grid lines.

Just as I had captured a model in my drawing because I fixated on relational thinking. So too can Tao Te Ching perhaps capture the world in its pages, we just need need to step back from the individual details. 

So therefore, 
\begin{quotation}
    When you understand all things can you step back from your own understanding? 
\end{quotation}
Is perhaps saying, can you step back and see the world in terms of their proper proportions? 

\section{Stillness and Flow}

What we have discussed so far may be regarded as the cosmic universe of the Tao defined in terms of relations. But relational thinking isn’t just how Tao Te Ching defines the world, but living within the world just as much as the world itself, and personally speaking this is where the utility of such becomes evident. For instance, do you ever feel lost? Have you, by your own intentions, muddied your own waters?

\begin{quotation}
    Who can be still until their mud settles and the water is cleared by itself?
    Can you remain tranquil until right action occurs by itself? 
\end{quotation}

This is where the aforementioned utility becomes obvious. Because you can forcefully intervene, but why go to the expense? If you have muddied your own waters, what use is it to further intervene? You will only make things all the more opaque. Rather, if you seek clarity, you must let things naturally disperse and dissipative into clarity by its own motions. Just as much as proper proportions manifest a more accurate picture of things. 

\section{Postscript}

\begin{quotation}
When the country falls into chaos, politicians talk about `patriotism'.
\end{quotation}

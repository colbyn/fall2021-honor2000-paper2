\section{Relationships and Proportions}

In Tao Te Ching the burden of understanding is placed on the reader rather than the author, and therefore just as the burden of understanding rests on the reader, so too, does the burden of translation rest on the translator especially. Therefore it may be exceeding difficult to discern meaning in Tao Te Ching. So much so, that at times it’s as though the author was conscious of such. 

\begin{quotation}\noindent
The Sages of old were profound and knew the ways of subtlety and discernment.
Their wisdom is beyond our comprehension.
Because their knowledge was so far superior I can only give a poor description.
\end{quotation}

(Which personally speaking should be the first line.)

But in some ways there can be personal gems buried deep within its pages. For instance,
\begin{quotation}\noindent
When you understand all things can you step back from your own understanding? 
\end{quotation}

The world depicted in Tao Te Ching is a world defined by mutual dichotomies and mutual interdependence, where something and nothing must necessarily and mutually give rise to each other, just as much as they give rise to themselves, and from this outgrowth, forms our reality.

\begin{quotation}\noindent
Look for it, and it can't be seen.
Listen for it, and it can't be heard.
Grasp for it, and it can't be caught.
These three cannot be further described, so we treat them as The One.
\\
Its highest is not bright.
Its depths are not dark.
\\
Unending, unnamable, it returns to nothingness.
Formless forms, and imageless images, subtle, beyond all understanding.
\\
Approach it and you will not see a beginning; follow it and there will be no end.
When we grasp the Tao of the ancient ones, we can use it to direct our life today.
To know the ancient origin of Tao: this is the beginning of wisdom.
\end{quotation}

When I was a kid I once drew something that was so good I never saw it until the end of the school year when I managed to get it back. From someone who isn’t necessarily known for being good at such, looking back I’ve come to the conclusion that what enabled me to draw such an exceeding good portrait was my unwitting fixation on relations. As I divided the paper into grids, I never fixated on the individual details themselves, but in terms of relations between the details and the grid lines. 

So therefore, 
\begin{quotation}\noindent
When you understand all things can you step back from your own understanding?
\end{quotation}
Is perhaps saying, can you step back and see the world in terms of their proper proportions? 

\section{Stillness and Flow}

What we have discussed so far may be regarded as the cosmic universe of the Tao defined in terms of relations. Yet in an individual context we see a rather different picture.

\begin{quotation}\noindent
Who can be still until their mud settles and the water is cleared by itself?
Can you remain tranquil until right action occurs by itself?
\end{quotation}

Essentially it follows the same principle, but in an individual context. If you have muddied your own waters, what use is it to further intervene? You will only make things all the more opaque. Rather, if you seek clarity, you must let things naturally disperse and dissipative into clarity by its own motions. Just as much as proper proportions manifest a more accurate picture of things. 








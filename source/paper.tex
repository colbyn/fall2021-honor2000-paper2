\section{The Axiomatic Foundation of Tao Te Ching}

You don’t know what you don’t know, and what you don’t know may itself be greater than what you know, and therefore you may not know anything -proportionally speaking- if you fixate solely on what you know. Which we could perhaps summarize: the more you know the less you know.

The aforementioned conclusion is paradoxical unlike the preceding rational, but being paradoxical doesn’t preclude information. Language is universal, but sometimes it’s not the best vehicle or medium for conveying information. For instance algebra began from language, but nowadays we use a better medium, a symbolic medium, for conveying mathematical structure.

In this manner it’s not obvious why e.g. sine and cosine are considered to be `odd' and `even' functions respectively. Yet the simple fact manifests itself so plainly by simply transforming the given function into a literal infinite series of terms, something that may itself be counterintuitive, yet there is a rationale to such. 

This is the analogy by which I propose we should try to understand Tao Te Ching. Similar to the position Tanaka takes in the limit of language in Daoism. Who wrote,

\begin{quotation}
The [Taoists] seem to have noticed that there is a limit to what language can do and that the limit of language is paradoxical.
\attrib{Tanaka}
\end{quotation}

Who also gives a similar analogy, 
\begin{quotation}
‘Don’t be guided by any guidance!’ is a paradoxical guidance. The only way to be guided by the guidance is by not being guided by it. But not being guided by the guidance, we are guided by it. How can we be guided or not be guided by this guidance?
\attrib{Tanaka}
\end{quotation}

Who likewise adopts a similar thesis,
\begin{quotation}
In general, the Western philosophers (and Indian philosophers) take language to be representative. Language is thought to represent the world, beliefs and so on (extra-linguistic ‘reality’). The Western study of language thus focuses on the semantics: how extra-linguistic reality is ‘mirrored’ in language. The main function of language for the Western philosophers is descriptive.
\attrib{Tanaka}
\end{quotation}

Yet Tanaka and I come to different conclusions, from a similar premise where we both argue that the manner in which something is conveyed may not necessary be the best for understanding such.

That is, Tanaka fixates on the paradoxical themes of Tao Te Ching and then asserts that such stems from a fixation on human behavior within society. But this doesn’t necessarily help us evaluate paradoxical information, let alone Tao Te Ching in general. 

Whereas I believe the more uniform and encompassing theme of Daoism is best understood by relational thinking. For instance, I have said that “the more you know the less you know”, is paradoxical. But in terms of relations, we are simply describing inversely proportional quantities

This is the unconventional manner or framework that I propose we should seek to understand Tao Te Ching. From the very first line,
\begin{verse}
The tao that can be described is not the eternal Tao.\\
\attrib{Chapter 1 | Translated by McDonald}
\end{verse}

For all possible descriptions of the Tao, there exists no description of the eternal Tao. In the very first line we are told that language itself is insufficient, by means of a relation between the descriptive Tao and the eternal Tao. Yet luckily, while it cannot be described, as I have just demonstrated, we can nevertheless derive information from the given statement, and therefore, just because something is paradoxical, doesn't preclude information.

This is the axiomatic foundation that we will build upon, just as euclidean geometry begins with several explicit axioms.


\section{Relationships and Proportions}

\begin{verse}
When you understand all things can you step back from your own understanding?\\
\attrib{Chapter 10 | Translated by McDonald}
\end{verse}

The world depicted in Tao Te Ching is a world defined by mutual dichotomies and mutual interdependence, where something and nothing must necessarily and mutually give rise to each other, just as much as they give rise to themselves, and from this outgrowth, forms our reality.

\begin{verse}
Look for it, and it can't be seen.\\
Listen for it, and it can't be heard.\\
Grasp for it, and it can't be caught.\\
These three cannot be further described, so we treat them as The One.\\

Its highest is not bright.\\
Its depths are not dark.\\

Unending, unnamable, it returns to nothingness.\\
Formless forms, and imageless images, subtle, beyond all understanding.\\

Approach it and you will not see a beginning; follow it and there will be no end.\\
When we grasp the Tao of the ancient ones, we can use it to direct our life today.\\
To know the ancient origin of Tao: this is the beginning of wisdom.\\
\attrib{Chapter 14 | Translated by McDonald}
\end{verse}

Just like any other model, you must describe how the model works, and in this manner, Tao Te Ching relationally models the world in the above manner. It circumvents certain discrete definitions in favor of indiscrete relations, and from these relations, it builds an axiomatic model that underpins the philosophy.

That is, this philosophy is given in terms of relations. Such as good and evil: good and evil mutually coexist, you cannot have one without the other. They imply and define each other just as much as they define themselves, and therefore good and evil aren’t standalone concepts, but relationally interdependent concepts.

Analogously, when I was a kid I once drew something that was so good I never saw it until the end of the school year when I managed to get it back. From someone who isn’t necessarily known for being good at such, looking back I’ve come to the conclusion that what enabled me to draw such an exceeding good portrait was my unwitting fixation on relations. As I divided the paper into grids, I never fixated on the individual details themselves, but in terms of relations between the details and the grid lines.

Just as I had captured a model in my drawing because I fixated on relational thinking. So too can Tao Te Ching perhaps capture the world in its pages, we just need need to step back from the individual details.

So therefore,
\begin{verse}
When you understand all things can you step back from your own understanding?\\
\attrib{Chapter 10 | Translated by McDonald}
\end{verse}

\newpage

Is perhaps saying, can you step back and see the world in terms of their proper proportions?

Likewise from the very same chapter, 

\begin{verse}
Nurture the darkness of your soul until you become whole.\\
\attrib{Chapter 10 | Translated by McDonald}
\end{verse}

Unlike western philosophy with its very singular fixations, in Daoism, as I have said, the world depicted in Tao Te Ching is a world defined by mutual dichotomies and mutual interdependence, and therefore, both mixtures of yourself depend on each other just as much as they individually depend on their own qualities. 

\begin{verse}
We mold clay into a pot, but it is the emptiness inside that makes the vessel useful.\\
We fashion wood for a house, but it is the emptiness inside that makes it livable.\\
We work with the substantial, but the emptiness is what we use.\\
\attrib{Chapter 11 | Translated by McDonald}
\end{verse}

\section{Stillness and Flow}

What we have discussed so far may be regarded as the cosmic universe of the Tao defined in terms of relations. But relational thinking isn’t just how Tao Te Ching defines the world, but living within the world just as much as the world itself, and personally speaking this is where the utility of such becomes evident. For instance, have you, by your own intentions, muddied your own waters?

\begin{verse}
Who can be still until their mud settles and the water is cleared by itself?\\
Can you remain tranquil until right action occurs by itself?\\
\attrib{Chapter 15 | Translated by McDonald}
\end{verse}

In a shallow stream there is a proportional relationship between action and clarity, where action diminishes clarity, and therefore, if you seek clarity, your actions must subside, you must let things naturally disperse and dissipative into clarity by its own motions. Likewise, the person of action must contend with murky waters. 

\begin{verse}
Five colors blind the eye.\\
Five notes deafen the ear.\\
Five flavors make the palate go stale.\\
Too much activity deranges the mind.\\
Too much wealth causes crime.\\
The Master acts on what she feels and not what she sees.\\
She shuns the latter, and prefers to seek the former.\\
\attrib{Chapter 12 | Translated by McDonald}
\end{verse}

I think a good analogy is that our sensory faculties evolved according to the pressures pertaining to natural selection which itself can be a slow process. Therefore, could the processes that evolved for very specific environments also mislead us in a world marred by change?

For instance some people think that allergies, something that seemingly has no utility, may therefore be a result of obsolete adaptions to very unsanitary conditions, and therefore if you are too clean (so to speak), your immune system may get confused because it lacks proportion, since it never had to contend with the bad, it may overreact when exposed to otherwise harmless substances, because again, it lacks proportion. Just as much as one who solely fixates on the good lacks certain proportions that may result in counterproductive outcomes. 

\begin{quotation}
Nurture the darkness of your soul until you become whole.
\attrib{Chapter 10 | Translated by McDonald}
\end{quotation}


\section{Change}

\begin{verse}
the Master can act without doing anything and teach without saying a word.\\
Things come her way and she does not stop them; things leave and she lets them go.\\
She has without possessing, and acts without any expectations.\\
When her work is done, she takes no credit.\\
That is why it will last forever.\\
\attrib{Chapter 2 | Translated by McDonald}
\end{verse}

I’d like to conclude with a larger theme that I think exists in Daoism, and this is,


\begin{verse}
If you want something to return to the source, you must first allow it to spread out.\\
If you want something to weaken, you must first allow it to become strong.\\
If you want something to be removed, you must first allow it to flourish.\\
If you want to possess something, you must first give it away.\\
This is called the subtle understanding of how things are meant to be.\\
The soft and pliable overcomes the hard and inflexible.\\
Just as fish remain hidden in deep waters, it is best to keep weapons out of sight.\\
\attrib{Chapter 36 | Translated by McDonald}
\end{verse}

In Daoism we don’t force outcomes to manifest themselves, but let things grow out by their own accord and by means of their own motions. Just as much as we let things naturally disperse and dissipative into clarity by its own motions, and therefore,

\begin{verse}
That which offers no resistance, overcomes the hardest substances. That which offers no resistance can enter where there is no space.\\
Few in the world can comprehend the teaching without words, or understand the value of non-action.\\
\attrib{Chapter 43 | Translated by McDonald}
\end{verse}





